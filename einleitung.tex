Das Unternehmen, die Funk, Zander \& Partner GmbH, hat sich seit der Gründung im 
Jahr 1992 auf die elektronische Datenverarbeitung und Unternehmensberatung spezialisiert.
Das weitere Unternehmensangebot umfasst anwenderbezogene Schulungen und Programmierungsarbeiten.
Die Auseinandersetzung mit dem Thema "'Entwicklung  eines  Erweiterungsmoduls  für 
die  Sage Office Line zum Druck von Etiketten nach \emph{GHS/CLP}\footnote{\label{foot:ghs}
siehe \nameref{sec:Grundlagen} auf Seite \pageref{sec:Grundlagen}} Verordnung 
sowie eines Etikettendesigners"' ist im Rahmen einer Kundenanforderung entstanden.

\subsection{Motivation}
\label{subsec:Motivation}

Für Unternehmen der chemischen Industrie ist es notwendig, aufgrund neuer
gesetzlicher Bestimmungen zur Einstufung und Kennzeichnung von Chemikalien,
Etiketten nach \emph{GHS/CLP}\textsuperscript{\ref{foot:ghs}} Verordnung zu nutzen.
Aus diesem Grund ist es erforderlich für die Sage Office Line ein neues Modul zum Druck 
dieser Art von Etiketten zu entwickeln. Die Sage Office Line ist eine 
ERP\footnote{\label{foot:erp}
Enterprise-Resource-Planning: Effizente Planung / Steuerung von Unternehmensressourcen. 
\cite{ERP}} Softwarelösung der Firma Sage Software GmbH, welche eine 
Tochterfirma der britischen Sage Group ist. "'Sage gehört zu einem der größten 
Anbieter von Softwarelösungen und Services für den deutschen Mittelstand."' \cite{sage} 


\subsection{Zielsetzung}
\label{subsec:Zielsetzung}

Das Ziel ist die Schaffung eines Moduls zum Druck von Etiketten nach 
\emph{GHS/CLP}\textsuperscript{\ref{foot:ghs}} Verordnung zur Erweiterung des Funktionsumfangs 
des ERP - Systems der Sage Office Line. Diese Funktionsweise muss, aufgrund der gesetzlichen
Bestimmungen, bis zum 01. Juni 2015 umgesetzt und funktionsbereit sein\cite{ifag}. 
Des Weiteren war die Umsetzung des Eitkettendesigners bis zum 30. Juni 2015 angedacht.  