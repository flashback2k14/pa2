Der Ausgangspunkt für die Auseinandersetzung mit dem Thema "'Entwicklung  eines  
Erweiterungsmoduls für die  Sage Office Line zum Druck von Etiketten nach \emph{GHS/CLP} 
Verordnung sowie eines Etikettendesigners"' war ein Kundenauftrag. Das Ziel des Kundenauftrags 
ist die Einführung einer automatisierten Lösung zur Umsetzung der neuen gesetzlichen 
Verpflichtung für die chemische Industrie. Daraus folgt die Etikettierung der Waren mittels der 
Schaffung eines Zusatzmoduls für die Sage Office Line.
\newline 
\newline
\noindent
Die Anbindung einer Erweiterung ist hinsichtlich des modularen Aufbaus der Sage Office Line 
möglich. Die verwendete Technologie in der Softwareentwicklung basiert auf dem Microsoft .NET 
Framework in Verbindung mit Microsoft Access und dem Microsoft COM Framework. Die Verwaltung der 
Daten wurde mit einem Microsoft SQL Server realisiert. 
\newline
\newline
\noindent
Das Ergebnis ist eine voll integrierte Erweiterung für die Sage Office Line, welche die 
gesetzlichen Voraussetzung zur Etikettierung erfüllt. Ein weiteres 
Resultat der Entwicklung ist die wiederverwendbare Logik zur Erstellung von Etiketten anhand
von Daten aus einer MS SQL Datenbank. Als zusätzlicher Einsatzort wäre die Erstellung von 
Adressetiketten für den Versand möglich. An dieser Stelle wäre die Schaffung eines Zusatzmoduls im Bereich 
Adressverwaltung / Kundenstammdaten denkbar. Des Weiteren soll perspektivisch die Entwicklung 
des ursprünglich geplanten Etikettendesigner als internes Projekt durchgeführt werden. Der 
Etikettendesigner soll sowohl innerhalb der Sage Office Line, wie auch ohne vorhandener Sage 
Office Line als eigenständige Softwarelösung nutzbar sein.

